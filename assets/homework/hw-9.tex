\documentclass[11pt]{article}

\usepackage{amsmath}
\usepackage{amssymb}
\usepackage{enumerate,comment}
\usepackage{url}
\usepackage{color}
\usepackage[margin=1.2in]{geometry}
\usepackage{fancyhdr}
\usepackage{xcolor}
\usepackage{hyperref}
\usepackage{cleveref}
\usepackage{mdframed}

\pagestyle{fancy}
\setlength{\headheight}{20pt}
\setlength{\headsep}{20pt}
\fancyhead[L]{\small{CS 276, Fall 2024}}
\fancyhead[R]{\small{Prof. Sanjam Garg}}

\newcommand\custombox[2]{%%
    \fbox{\rule{#1}{0pt}\rule[-0.5ex]{0pt}{#2}}}

\newcommand\answerbox{%%
    \fbox{\rule{1in}{0pt}\rule[-0.5ex]{0pt}{4ex}}}

\newtheorem{theorem}{Theorem}[section]
\newtheorem{definition}[theorem]{Definition}
\newtheorem{corollary}[theorem]{Corollary}
\newtheorem{lemma}[theorem]{Lemma}
\newtheorem{claim}[theorem]{Claim}
\newtheorem{fact}[theorem]{Fact}
\newtheorem{conjecture}[theorem]{Conjecture}
\newtheorem{assumption}[theorem]{Assumption}
\newenvironment{solution}{\color{blue}\noindent{\bf Solution}\hspace*{1em}}{\qed\medskip}
\newcommand{\proof}{\noindent{\bf Proof. }} %% To begin a proof write \proof
\newcommand{\qed}{\mbox{}\hspace*{\fill}\nolinebreak\mbox{$\rule{0.6em}{0.6em}$}} %%to end your proof write $\qed$.

\numberwithin{equation}{section}

% mathbf
\newcommand{\bfA}{\mathbf{A}}
\newcommand{\bfa}{\mathbf{a}}
\newcommand{\bfb}{\mathbf{b}}
\newcommand{\bfc}{\mathbf{c}}
\newcommand{\bfe}{\mathbf{e}}
\newcommand{\bfI}{\mathbf{I}}
\newcommand{\bfM}{\mathbf{M}}
\newcommand{\bfm}{\mathbf{m}}
\newcommand{\bfp}{\mathbf{p}}
\newcommand{\bfr}{\mathbf{r}}
\newcommand{\bfs}{\mathbf{s}}
\newcommand{\bfT}{\mathbf{T}}
\newcommand{\bfu}{\mathbf{u}}
\newcommand{\bfv}{\mathbf{v}}
\newcommand{\bfx}{\mathbf{x}}
\newcommand{\bfy}{\mathbf{y}}

% mathbb
\newcommand{\bbE}{\mathbb{E}}
\newcommand{\bbF}{\mathbb{F}}
\newcommand{\bbG}{\mathbb{G}}
\newcommand{\bbN}{\mathbb{N}}
\newcommand{\bbZ}{\mathbb{Z}}

% mathcal
\newcommand{\cA}{\mathcal{A}}
\newcommand{\cB}{\mathcal{B}}
\newcommand{\cC}{\mathcal{C}}
\newcommand{\cD}{\mathcal{D}}
\newcommand{\cF}{\mathcal{F}}
\newcommand{\cG}{\mathcal{G}}
\newcommand{\cH}{\mathcal{H}}
\newcommand{\cK}{\mathcal{K}}
\newcommand{\cM}{\mathcal{M}}
\newcommand{\cQ}{\mathcal{Q}}
\newcommand{\cR}{\mathcal{R}}
\newcommand{\cS}{\mathcal{S}}
\newcommand{\cX}{\mathcal{X}}
\newcommand{\cY}{\mathcal{Y}}

% Crypto terms
\newcommand{\Setup}{\mathsf{Setup}}
\newcommand{\Gen}{\mathsf{Gen}}
\newcommand{\Enc}{\mathsf{Enc}}
\newcommand{\Dec}{\mathsf{Dec}}
\newcommand{\gen}{\mathsf{Gen}}
\newcommand{\enc}{\mathsf{Enc}}
\newcommand{\dec}{\mathsf{Dec}}
\newcommand{\Sign}{\mathsf{Sign}}
\newcommand{\sign}{\mathsf{Sign}}
\newcommand{\Prove}{\mathsf{Prove}}
\newcommand{\Verify}{\mathsf{Verify}}
\newcommand{\verify}{\mathsf{Verify}}
\newcommand{\RO}{\mathsf{RO}}
\newcommand{\MAC}{\mathsf{MAC}}
\newcommand{\mac}{\mathsf{Mac}}
\newcommand{\pk}{\mathsf{pk}}
\newcommand{\sk}{\mathsf{sk}}
\newcommand{\mpk}{\mathsf{mpk}}
\newcommand{\msk}{\mathsf{msk}}
\newcommand{\ct}{\mathsf{ct}}
\newcommand{\secp}{\lambda}

\newcommand{\Rank}{\mathsf{Rank}}
\newcommand{\GF}{\mathsf{GF}}
\newcommand{\msg}{\mathsf{msg}}
\newcommand{\negl}{\mathsf{negl}}
\newcommand{\nonnegl}{\mathsf{nonnegl}}
\newcommand{\st}{\mathsf{st}}
\newcommand{\adv}{\cA}
\newcommand{\hyb}{\mathsf{Hyb}}
\newcommand{\poly}{\mathsf{poly}}

% Math Notation
\newcommand{\getsr}{\stackrel{\$}{\gets}}
\newcommand{\bin}{\{0,1\}}
\newcommand{\bit}{\bin}

\newcommand{\duedate}{Friday November 22nd, 2024 at 8:59pm via Gradescope}

\begin{document}
\section*{CS 276: Homework 9\\ {\small Due Date: \duedate} }

\section{Simulation-Sound NIZKs}
We will use the Fiat-Shamir transform to convert the interactive sigma protocol from homework 8 into a non-interactive zero-knowledge proof (NIZK). 

We will also define the notion of simulation soundness for NIZKs, which combines soundness and zero-knowledge into one security definition. Simulation soundness essentially states that an adversary who sees simulated proofs of true and false statements of their choosing, cannot produce an accepting proof on a different false statement. 

Simulation-sound NIZKs can be used to construct CCA2-secure encryption and signatures, among other applications.

\paragraph{The Fiat-Shamir Transform:}
Let us start with the sigma protocol from homework 8 and make it non-interactive by computing the verifier's message $m$ with a random oracle $\cH$ applied to the partial transcript of the protocol. This is known as the \textit{Fiat-Shamir transform}.\newline

As in homework 8, let $\bbG$ be a cryptographic group of prime order $p$, where $\frac{1}{p} = \negl(\secp)$. Let $d_{in}, d_{out} \in \bbN$ be the dimensions of the input and output spaces, respectively. A function $F$ mapping $\bbZ_p^{d_{in}} \to \bbG^{d_{out}}$ is \textit{homomorphic} if for any $\bfx, \bfx' \in \bbZ_p^{d_{in}}$, $F(\bfx + \bfx') = F(\bfx) \cdot F(\bfx')$. An \textit{instance} of the language $L$ is any tuple $(F, \bfy)$ such that $F$ is a homomorphic function mapping $\bbZ_p^{d_{in}} \to \bbG^{d_{out}}$, and $\bfy \in \mathsf{Im}(F)$. The corresponding \textit{witness} is an input $\bfx \in \bbZ_p^{d_{in}}$ such that $F(\bfx) = \bfy$.

Additionally, let us assume that if we sample $\bfx \getsr \bbZ_p^{d_{in}}$, then $F(\bfx)$ has min-entropy $\omega(\log^2(\secp))$. In other words, for any $\bfy \in \bbG^{d_{out}}$,
\[\Pr_{\bfx \getsr \bbZ_p^{d_{in}}}[F(\bfx) = \bfy] \leq 2^{- \omega(\log^2(\secp))} = \negl(\secp)\]

Also, let $\cH$ be a random oracle mapping $\bbG^{d_{out}} \times \bbG^{d_{out}} \to \bbZ_p$. Finally, the NIZK is a pair of algorithms $(\Prove, \Verify)$, which are constructed as follows.\newline

\noindent\underline{$\Prove(\bfx, \bfy)$:}
\begin{enumerate}
    \item Sample $\bfa \getsr \bbZ_p^{d_{in}}$, and compute $\bfb = F(\bfa)$.
    \item Compute $m = \cH(\bfy, \bfb)$.
    \item Compute $\bfc = m \cdot \bfx + \bfa$ and output $\pi = (\bfb, \bfc)$.
\end{enumerate}

\noindent\underline{$\Verify(\bfy, \pi)$:}
\begin{enumerate}
    \item Compute $m = \cH(\bfy, \bfb)$.
    \item If $F(\bfc) = \bfy^m \cdot \bfb$, then output $\mathsf{accept}$. Else output $\mathsf{reject}$.
\end{enumerate}

\paragraph{Zero-Knowledge:}
Let us define the notion of zero-knowledge for NIZKs. 

\begin{definition}[Zero-Knowledge Adversary and Simulator]
    The zero-knowledge adversary $\cA$ is run in one of the following games, $\cG_{\mathsf{Real}}$ or $\cG_{\mathsf{Ideal}}$, and they are not told which one. $\cA$ makes \textit{proof queries} of the form $(\bfx, \bfy) \in \bbZ_p^{d_{in}} \times \bbG^{d_{out}}$, where $F(\bfx) = \bfy$, and \textit{random oracle queries} of the form $(\bfy, \bfb) \in \bbG^{d_{out}} \times \bbG^{d_{out}}$, and finally they output a bit $b$ in order to guess which game they are in.
    
    In the \textit{real world}, $\cG_{\mathsf{Real}}$, the challenger samples a random oracle $\cH$ and responds to each random oracle query with $\cH(\bfy, \bfb)$. For each proof query $(\bfx, \bfy)$, the challenger responds with $\pi = \Prove(\bfx, \bfy)$.

    In the \textit{ideal world}, $\cG_{\mathsf{Ideal}}$, there is a PPT simulator $\cS$ that handles the queries. $\cS$ receives each random oracle query $(\bfy, \bfb)$ and computes the response $\cS.\RO(\bfy, \bfb)$. For each proof query, $(\bfx, \bfy)$ such that $F(\bfx) = \bfy$, $\cS$ only receives $\bfy$ and must compute the response $\cS.\Prove(\bfy)$.
\end{definition}

\begin{definition}[Zero-Knowledge for NIZKs]\label{def:zk-for-nizks}
    The NIZK satisfies \textbf{zero-knowledge} if there exists a PPT simulator $\cS$ such that for all PPT adversaries $\cA$, 
    \[\left|\Pr[\cA \to 1 \text{ in } \cG_{\mathsf{Real}}] - \Pr[\cA \to 1 \text{ in } \cG_{\mathsf{Ideal}}]\right| = \negl(\secp)\]
\end{definition}

\paragraph{Simulation Soundness:} In the definition of zero-knowledge, $\cS$ is only required to output an accepting proof for a statement in the language (i.e. an $(\bfx, \bfy)$ for which $F(\bfx) = \bfy$). Simulation soundness allows the adversary to run $\cS$ on false statements as well (where $\bfy \notin \mathsf{Im}(F)$) and guarantees that the adversary cannot forge an accepting proof on a new false statement.

\begin{definition}[Simulation Soundness Game $\cG_{\mathsf{SS}}$]
    The simulation soundness adversary $\cB$ interacts with $\cS$ directly. $\cB$ can make proof queries of the form $\bfy \in \bbG^{d_{out}}$ and receives the response $\cS.\Prove(\bfy)$. $\cB$ can also make random oracle queries of the form $(\bfy, \bfb) \in \bbG^{d_{out}} \times \bbG^{d_{out}}$ and receives the response $\cS.\RO(\bfy, \bfb)$.

    Finally $\cB$ outputs a statement-proof tuple $(\bfy^*, \pi^*)$, which the challenger verifies by computing $\Verify(\bfy^*, \pi^*)$. If $\Verify$ queries the random oracle, the challenger queries $\cS.\RO$. 
    
    $\cB$ wins $\cG_{\mathsf{SS}}$ if $(\bfy^*, \pi^*)$ was not a previous query-response pair for $\cS.\Prove$, and $\Verify(\bfy^*, \pi^*)$ outputs $\mathsf{accept}$, and $\bfy \notin \mathsf{Im}(F)$ ($\bfy$ is a false statement).
\end{definition}

\begin{definition}[Simulation Soundness]
    A NIZK is simulation-sound if there exists a PPT simulator $\cS$ such that the following hold:
    \begin{itemize}
        \item \textit{Zero Knowledge:} For all PPT zero-knowledge adversaries $\cA$,
        \[\left|\Pr[\cA \to 1 \text{ in } \cG_{\mathsf{Real}}] - \Pr[\cA \to 1 \text{ in } \cG_{\mathsf{Ideal}}]\right| = \negl(\secp)\]
        \item \textit{Unforgeability:} For all PPT simulation soundness adversaries $\cB$,
        \[\Pr[\cB \text{ wins } \cG_{\mathsf{SS}}] = \negl(\secp)\]
    \end{itemize}
\end{definition}

\paragraph{Question:} Prove that the NIZK $(\Prove, \Verify)$ constructed above satisfies simulation soundness.

\vspace{5mm}
\begin{solution}
TBD
\end{solution}

\end{document}