\documentclass[11pt]{article}

\usepackage{amsmath}
\usepackage{amssymb}
\usepackage{enumerate,comment}
\usepackage{url}
\usepackage{color}
\usepackage[margin=1.2in]{geometry}
\usepackage{fancyhdr}
\usepackage{xcolor}
\usepackage{hyperref}
\usepackage{cleveref}
\usepackage{mdframed}

\pagestyle{fancy}
\setlength{\headheight}{20pt}
\setlength{\headsep}{20pt}
\fancyhead[L]{\small{CS 276, Fall 2024}}
\fancyhead[R]{\small{Prof. Sanjam Garg}}

\newcommand\custombox[2]{%%
    \fbox{\rule{#1}{0pt}\rule[-0.5ex]{0pt}{#2}}}

\newcommand\answerbox{%%
    \fbox{\rule{1in}{0pt}\rule[-0.5ex]{0pt}{4ex}}}

\newtheorem{theorem}{Theorem}[section]
\newtheorem{definition}[theorem]{Definition}
\newtheorem{corollary}[theorem]{Corollary}
\newtheorem{lemma}[theorem]{Lemma}
\newtheorem{claim}[theorem]{Claim}
\newtheorem{fact}[theorem]{Fact}
\newtheorem{conjecture}[theorem]{Conjecture}
\newtheorem{assumption}[theorem]{Assumption}
\newenvironment{solution}{\color{blue}\noindent{\bf Solution}\hspace*{1em}}{\qed\medskip}
\newcommand{\proof}{\noindent{\bf Proof. }} %% To begin a proof write \proof
\newcommand{\qed}{\mbox{}\hspace*{\fill}\nolinebreak\mbox{$\rule{0.6em}{0.6em}$}} %%to end your proof write $\qed$.

\numberwithin{equation}{section}

% Crypto terms
\newcommand{\Gen}{\mathsf{Gen}}
\newcommand{\Enc}{\mathsf{Enc}}
\newcommand{\Dec}{\mathsf{Dec}}
\newcommand{\gen}{\mathsf{Gen}}
\newcommand{\enc}{\mathsf{Enc}}
\newcommand{\dec}{\mathsf{Dec}}
\newcommand{\Sign}{\mathsf{Sign}}
\newcommand{\sign}{\mathsf{Sign}}
\newcommand{\Verify}{\mathsf{Verify}}
\newcommand{\verify}{\mathsf{Verify}}
\newcommand{\MAC}{\mathsf{MAC}}
\newcommand{\mac}{\mathsf{Mac}}
\newcommand{\pk}{\mathsf{pk}}
\newcommand{\sk}{\mathsf{sk}}

% mathbb
\newcommand{\bbE}{\mathbb{E}}
\newcommand{\bbF}{\mathbb{F}}
\newcommand{\bbG}{\mathbb{G}}
\newcommand{\bbZ}{\mathbb{Z}}

% mathcal
\newcommand{\cA}{\mathcal{A}}
\newcommand{\cB}{\mathcal{B}}
\newcommand{\cC}{\mathcal{C}}
\newcommand{\cD}{\mathcal{D}}
\newcommand{\cF}{\mathcal{F}}
\newcommand{\cG}{\mathcal{G}}
\newcommand{\cH}{\mathcal{H}}
\newcommand{\cK}{\mathcal{K}}
\newcommand{\cM}{\mathcal{M}}
\newcommand{\cQ}{\mathcal{Q}}
\newcommand{\cR}{\mathcal{R}}
\newcommand{\cS}{\mathcal{S}}
\newcommand{\cX}{\mathcal{X}}
\newcommand{\cY}{\mathcal{Y}}

% Words
\newcommand{\Rank}{\mathsf{Rank}}
\newcommand{\GF}{\mathsf{GF}}
\newcommand{\msg}{\mathsf{msg}}
\newcommand{\negl}{\mathsf{negl}}
\newcommand{\nonnegl}{\mathsf{nonnegl}}
\newcommand{\st}{\mathsf{st}}
\newcommand{\adv}{\cA}
\newcommand{\hyb}{\mathsf{Hyb}}

% Math Notation
\newcommand{\getsr}{\stackrel{\$}{\gets}}
\newcommand{\bin}{\{0,1\}}
\newcommand{\bit}{\bin}

\newcommand{\duedate}{Friday September 27th, 2024 at 8:59pm via Gradescope}

\newif\ifsol
\soltrue

\begin{document}
\section*{CS 276: Homework 4\\ {\small Due Date: \duedate} }


\section{Carter-Wegman Message Authentication Code}

The Carter-Wegman MAC is built from a PRF and a hash function as follows. Let $p$ be a large prime. Let $n$ be the security parameter. Let $F: \cK_F \times \bit^n \to \bbZ_p$ be a secure PRF, and let $H:\cK_H \times \cM \to \bbZ_p$ be a hash function. Next:
\begin{enumerate}
    \item $\MAC$ takes a key $(k_H, k_F) \in \cK_H \times \cK_F$ and a message $m \in \cM$. Then $\MAC$ samples $r \getsr \bit^n$ and computes:
    \[v = H(k_H, m) + F(k_F, r)\]
    Finally $\MAC$ outputs $(r,v)$.
    \item $\Verify$ takes a key $(k_H, k_F) \in \cK_H \times \cK_F$, a message $m \in \cM$, and a tag $(r,v) \in \bit^n \times \bbZ_p$. Then $\Verify$ checks that $v = H(k_H, m) + F(k_F, r)$. If so, $\Verify$ outputs $1$ (accept). If not, $\Verify$ outputs $0$ (reject).
\end{enumerate}

Now we will consider two possible choices for $H$:
\begin{enumerate}
    \item $H_1$ takes a key $k_H \getsr \bbZ_p$ and an input $m = (m_1, \dots, m_\ell) \in \bbZ_p^\ell$, where $\ell$ is polynomial in $n$. Then
    \[H_1(k_H, m) = k_H^{\ell} + \sum_{i = 1}^\ell k_H^{\ell-i} \cdot m_i\]
    \item $H_2(k_H, m) = k_H \cdot H_1(k_H, m)$
\end{enumerate}

\paragraph{Question:} Prove that the Carter-Wegman MAC is insecure if it is constructed with $H = H_1$, but it is secure if it is constructed with $H = H_2$.\newline

\noindent The following definition of MAC security will be useful.
\begin{definition}[MAC Security]
A MAC is secure if for any non-uniform PPT adversary $\cA$, 
\[\Pr[\mathsf{MAC}\text{-}\mathsf{Forge}_{\cA}(n) \to 1] \leq \negl(n)\]

\noindent\underline{$\mathsf{MAC}\text{-}\mathsf{Forge}_{\cA}(n)$}:
    \begin{enumerate}
        \item \textbf{Setup:} The challenger samples $k$ uniformly from the key space. $\cA$ is given $1^n$.
        \item \textbf{Query:} The adversary submits a message $m^{(i)}$; then the challenger computes a tag $t^{(i)} \gets \MAC(k, m^{(i)})$ and sends it to the adversary. The adversary may submit any polynomial number of message queries.
        
        Let $\cQ = \{(m^{(1)}, t^{(1)}), \dots, (m^{(q)}, t^{(q)})\}$ be the set of messages $m^{(i)}$ submitted in the query phase along with the tags $t^{(i)}$ computed by $\MAC$.
        \item \textbf{Forgery:} The adversary outputs a message-tag pair $(m^*, t^*)$. The output of the game is $1$ if $(m^*,t^*) \notin \cQ$ and $\Verify(k, m^*, t^*) = 1$. The output is $0$ otherwise.
    \end{enumerate}
\end{definition}


\ifsol
\pagebreak
\begin{solution}
TBD    
\end{solution}
\fi

\end{document}